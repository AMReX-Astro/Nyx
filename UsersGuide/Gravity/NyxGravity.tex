\label{chap:Gravity}

In \nyx\ we always compute gravity by solving a Poisson equation on the mesh hierarchy.
To make sure this option is chosen correctly, we must always set \\

\noindent {\bf USE\_GRAV = TRUE} \\

\noindent in the GNUmakefile and \\

\noindent {\bf castro.do\_grav} = 1  \\
\noindent {\bf gravity.gravity\_type = PoissonGrav} \\

\noindent in the inputs file.  \\

To define the gravitational vector we set
\begin{equation}
\mathbf{g}(\mathbf{x},t) = -\nabla \phi 
\end{equation}
where 
\begin{equation}
\mathbf{\Delta} \phi = \frac{4 \pi G}{a} (\rho - \overline{\rho}) \label{eq:Self Gravity}
\end{equation}

\noindent where $\overline{\rho}$ is the average of $\rho$ over the entire domain if we assume triply periodic boundary conditions,
and $a(t)$ is the scale of the universe as a function of time.

