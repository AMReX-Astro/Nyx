\label{chap:HeatCool}

\nyx\ provides the capability to compute local heating and cooling effects due to radiation.
The motivation and algorithm for the heating and cooling components is documented in \cite{lukic15}, and the relevant code is located in the \texttt{Source/HeatCool} subdirectory.
The code is activated through the \texttt{USE\_HEATCOOL=TRUE} option in the \texttt{GNUmakefile}.
Mathematically, the heating and cooling can be described by a single ODE in each cell, to be integrated per time step $\Delta t$.
This ODE exhibits a sensitive relationship to quantities such as temperature and free electron density, and consequently it often requires sophisticated integration techniques to compute correctly.

\nyx\ provides a few different techniques for solving this ODE, which are selected via the \texttt{nyx.heat\_cool\_type} input parameter.
One method is to use the VODE ODE solver (selected with \texttt{nyx.heat\_cool\_type=3}).
The source code for VODE is included in the \texttt{Util/VODE} subdirectory and is compiled automatically with the rest of \nyx.
However, while VODE is sufficient for computing this ODE correctly, it is an old Fortran code which is no longer maintained, and consequently will not easily be adapted to future high-performance computing architectures.

VODE's successor is CVODE, which is a translation of the original VODE solver from Fortran to C.
CVODE is actively developed and maintained, and is more likely to be adapted to future architectures.
To use CVODE in \nyx, one may use the \texttt{nyx.heat\_cool\_type=5} input parameter.
Currently the performance of VODE is slightly better because CVODE evaluates the ODE RHS one more time than VODE per coarse time step integration.
Users should note that, while the VODE solver is compiled automatically in \nyx, CVODE must be compiled as a separate library; instructions for compiling CVODE are provided in the \amrex\ User Guide.
To link the external CVODE solver into \nyx, one must set \texttt{USE\_HEATCOOL=TRUE} as well as \texttt{USE\_CVODE=TRUE} in the \texttt{GNUmakefile}.

Finally, a third ODE integration option (which is new and \emph{\textbf{highly experimental}}) consists of using CVODE while treating groups of ODEs in different cells as a single system of coupled ODEs.
This option can be selected with the \texttt{nyx.heat\_cool\_type=7} option.
The purpose of this approach is to enable the evaluation of multiple RHSs simultaneously, using SIMD instructions.
SIMD parallelism comprises a large fraction of compute performance on modern HPC architectures, and consequently, this approach can lead to a significant performance gain in the ODE integration (which is the most expensive computational kernel in \nyx).
The number of ODEs (cells) which are computed simultaneously is chosen through the input parameter \texttt{nyx.simd\_width}.
On Intel Xeon Phi, with 512 bit-wide SIMD instructions, an appropriate value for this parameter might be 8 or 16, or perhaps larger; the value which yields the highest performance will vary by architecture.
However, users are cautioned that this mode remains \emph{\textbf{experimental}} and its results have not been subjected to the same level of verification as the other solver methods.
In particular, the are three numerical tolerances, available as input parameters, which affect the convergence of the scalar vs SIMD ODE integration:

\begin{itemize}
  \item \texttt{nyx.eos\_nr\_eps}: this is the convergence criterion for the Newton-Raphson iteration which is used to evaluate the ODE RHS
  \item \texttt{nyx.vode\_rtol}: this is the relative tolerance required for the ODE integration in VODE or CVODE
  \item \texttt{nyx.vode\_atol\_scaled}: this is the absolute tolerance required for the ODE integration in VODE or CVODE, scaled by the initial value of the independent variable in the ODE
\end{itemize}

These variables, in particular \texttt{nyx.vode\_rtol}, have different effects depending on whether one is integrating a single ODE at a time, or a system of ODEs simultaneously.
One should be mindful of the numerical differences which arise from these, which can be observed with the \texttt{fcompare} tool in \amrex.
