\section{The AGN Model}

% https://www.overleaf.com/9080833cntqkdydhkdq#/32617999/

\newcommand{\Msun}{M_\odot}

In the AGN model, super-massive black hole (SMBH) particles are formed
at {\em haloes}, where
each halo is defined by a connected mass enclosed by a
user-defined density isocontour.
In order to find haloes,
we use the Reeber package described in Section~\ref{sec:Reeber}.
Each AGN particle has the standard dark matter particle attributes of
position, velocity, and mass, as well as two additional attributes,
its stored accretion energy and its mass accretion rate.

\begin{table}
\begin{center}
  \caption{Parameters of the AGN model}

\begin{tabular}{l l l l}
    \toprule
In ``probin'' file & Parameter & Fiducial value & Explanation\\
    \midrule
* & $M_{\rm h, min}$ & $10^{10}~ \Msun$ & Minimum halo mass for SMBH placement\\
* & $M_{\rm seed}$ & $10^5~ \Msun$ & Seed mass of SMBH\\
% $n_{\rm local}$ & 8 & Number of cells in the ``local environment'', here CIC\\
{\tt T\_min} & $T_{\rm min}$ & $10^7$ K & Minimum heating of the surrounding gas\\
% $\chi$ & 0.01 & Two-mode change criteria (set to 0 if no momentum feedback) \\
{\tt bondi\_boost} & $\alpha$ & 100 & Bondi accretion boost factor \\      
{\tt max\_frac\_removed} & $f_{\rm max, removed}$ & 0.5 & Maximum fraction of mass removed from gas \\
{\tt eps\_rad} & $\epsilon_{\rm r}$ & 0.1 & Radiation efficiency\\
{\tt eps\_coupling} & $\epsilon{\rm c}$ & 0.15 & Coupling efficiency\\
{\tt eps\_kinetic} & $\epsilon_{\rm kin}$ & 0.1 & Kinetic feedback efficiency \\
{\tt frac\_kinetic} & $f_{\rm kin}$ & 0 & Fraction of feedback energy that is kinetic \\

\bottomrule
\end{tabular}

% \medskip
% Note that $n_{\rm local}$ is not really a parameter, but there is a
% method of how we define ``local environment''.  This is harder to
% change, as it is hard-coded in relevant routines, but is still listed
% here for enhanced transparency -- this is something we can \emph{in
%   principle} experiment with.  

  \label{tab:agn_params1}
 \end{center}

\medskip
* $M_{\rm h, min}$ and $M_{\rm seed}$ are not set in the ``probin''
file, but in the inputs file, by respectively
{\tt Nyx.mass\_halo\_min} and {\tt Nyx.mass\_seed}.


\end{table}


\subsection{Creating AGN Particles from Haloes}

% We use the Reeber package to find haloes, and then create super-massive
% black holes (SMBH) using the halo information.

Each halo with threshold mass of $M_h \geqslant M_{\rm h, min}$
that does not already host a black hole particle is seeded with
a black hole of mass $M_{\rm seed}$.
The initial position of this AGN particle is
the center of the cell where the density is highest in the halo.

When an AGN particle is created,
the density in its cell is reduced by the amount required for
mass to be conserved, and
the velocity of the AGN particle is initialized so that
momentum is conserved.
The accretion energy and mass accretion rate are initialized to zero.

\subsection{Merging AGN Particles}

Two AGN particles merge when both of these conditions obtain:
\begin{enumerate}
\item 
The distance between them, $l$, is less than the mesh spacing, $h$.
\item
\label{velocity-item}
The difference of their velocities, $v_{\rm rel}$,
is less than the circular velocity at distance $l$:
$$ v_{\rm rel} < \sqrt{GM_{\rm BH}/l}$$
where $M_{\rm BH}$ is the mass of the more massive SMBH in the pair,
and $G$ is the gravitational constant.
\end{enumerate}
Criterion~\ref{velocity-item} above
is necessary in order to prevent AGN particles from merging during
a fly-through encounter of two haloes, as this could lead to AGN particles
being quickly removed from the host halo due to momentum conservation.

The merger of two AGN particles is implemented as the less massive one
being removed, and its mass and momentum being transferred to the
more massive one.
% What about its stored accretion energy?

\subsection{Accretion}

For an AGN particle of mass $M_{\rm BH}$,
the Bondi--Hoyle accretion rate is
\begin{equation}
\dot{M}_{\rm B} = \alpha
\frac{4 \pi G^2 M_{\rm BH}^2 \overline{\rho}}{(\overline{c_s^2} + \overline{u^2})^{3/2}} ,
\end{equation}
where 
% $\alpha$ is the Bondi boost factor, and
$\overline{\rho}$, $\overline{c_s^2}$, and $\overline{u^2}$ are volume
averages with a cloud-in-cell stencil of the gas's density, 
squared sound speed, and squared velocity, respectively,
in the neighborhood of the particle.

The maximum black hole accretion rate is the Eddington limit,
\begin{equation}
\dot{M}_{\rm Edd} = 
\frac{4 \pi G M_{\rm BH} m_{\rm p}}{\epsilon_{\rm r} \sigma_{\rm T} c} \, ,
\end{equation}
with proton mass $m_{\rm p}$,
Thomson cross section $\sigma_{\rm T}$,
and speed of light $c$.
% , and $\epsilon_{\rm r}$ is the radiation efficiency parameter

The mass accretion rate of the SMBH is the smaller of the two rates above:
$\dot{M}_{\rm acc} = {\rm min} \{ \dot{M}_{\rm B}, \dot{M}_{\rm Edd} \}$.
Then the gas will lose mass $\dot{M}_{\rm acc} \Delta t$, where
$\Delta t$ is the length of the time step.
However, $\dot{M}_{\rm acc}$ is adjusted downward if necessary so that
when cloud-in-cell stencil weights are applied in the neighborhood of
the particle, the fraction of gas density removed from any cell of the
stencil is at most $f_{\rm max, removed}$.

The mass of the AGN particle increases by 
$(1-\epsilon_{\rm r}) \dot{M}_{\rm acc} \Delta t$,
while $\dot{M}_{\rm acc} \Delta t$ amount of gas mass is removed from
the grid according to cloud-in-cell stencil weights in the
neighborhood of the particle.
The momentum transfer can be computed by assuming the velocity of the
gas is unchanged;  thus the gas in each cell loses momentum in
proportion to its mass loss, and the particle gains the sum of the gas
momentum loss multiplied by $(1-\epsilon_{\rm r})$. 

\subsection{Feedback Energy}

Feedback energy is stored in an AGN particle variable $E_{\rm AGN}$,
and is accumulated over time until released.
The fraction $f_{\rm kin}$ goes to kinetic energy, and the rest to
thermal energy.

\subsubsection{Thermal Feedback}

% In high accretion rates,
We increment $E_{\rm AGN}$ by
thermal feedback energy, calculated from the
mass accretion rate as 
\begin{equation} % high energy
\Delta E_{\rm thermal} = (1 - f_{\rm kin})
\epsilon_{\rm c} \epsilon_{\rm r} \dot{M}_{\rm acc} c^2 \Delta t .
\end{equation}
% where $\epsilon{\rm c}$ is the coupling efficiency.
% This energy is not released immediately to the surrounding gas, but
% is instead accumulated over many time steps and stored in an AGN
% particle variable $E_{\rm AGN}$.

\subsubsection{Kinetic/Momentum Feedback}

We increment $E_{\rm AGN}$ by the kinetic feedback energy
\begin{equation} % low energy
\Delta E_{\rm kinetic} =
f_{\rm kin} \epsilon_{\rm kin} \dot{M}_{\rm acc} c^2 \Delta t .
\end{equation}
We also need to adjust the energy density
and momentum density of the gas.
We do this by computing a jet velocity
\begin{equation}
\vec{v}_{\rm jet} = \sqrt{\frac{2 \Delta E_{\rm kinetic}}{m_{\rm g}}} \vec{n}
\end{equation}
where $m_{\rm g}$ is the total gas mass inside the cloud-in-cell local
environment, and $\vec{n}$ is a {\em randomly} chosen unit vector.
We add
$\rho \vec{v}$ to the momentum density $\vec{p}$ of the gas,
and
$\vec{v}_{\rm jet} \cdot \vec{p}$ to its energy density,
both of these weighted by
the cloud-in-cell stencil of the particle.

\subsubsection{Releasing Feedback Energy}

The accumulated energy is released when
\begin{equation}
E_{\rm AGN} > m_{\rm g} \overline{e}
\label{eq:E_agn}
\end{equation}
where $\overline{e}$ is the average specific internal energy of the
gas over the cloud-in-cell stencil, obtained from the equation of
state using temperature $T_{\rm min}$ and
average density of the gas over the same stencil,
and $m_{\rm g}$ is the total gas mass inside the cloud-in-cell local environment.
% where $k_{\rm B}$ is Boltzmann's constant, $m_{\rm p}$ is proton mass,
% $\mu$ is the 
% mean mass per particle in the gas, $\gamma$ is the ratio of specific
% heats, and $m_{\rm g}$ is the total gas mass inside the cloud-in-cell
% local environment.
% Note that internal energy is $e = k_{\rm B} T / (\gamma -1) \mu m_{\rm p}$, thus
% the most consistent way to implement equation \ref{eq:E_agn} in the
% code is probably by calling the appropriate equation of state routine
% with $T = T_{\rm min}$. 


% \section{Finding Haloes with Reeber}

\section{The Reeber Package}
\label{sec:Reeber}

Reeber is a separate package with a halo finder.
Here are the Reeber parameters that are assigned in the input file.

\begin{table*}[h]
\begin{scriptsize}
\begin{tabular}{|l|l|l|l|} \hline
Parameter & Definition & Acceptable Values & Default \\
\hline
{\bf reeber.halo\_int} & timesteps between halo finder calls &
Integer & -1 (none) \\
{\bf reeber.negate} & allow negative values for analysis &
0 if false, 1 if true & 1 \\
{\bf reeber.halo\_density\_vars} & density variable list & density, particle\_mass\_density & ``density'' \\
{\bf reeber.halo\_extrema\_threshold} & extrema threshold for haloes & Real  & 200. \\
{\bf reeber.halo\_component\_threshold} & component threshold for haloes & Real & 82. \\
{\bf reeber.absolute\_halo\_thresholds} & are halo thresholds absolute &
0 if multiples of mean, 1 if absolute & 0 \\
\hline
\end{tabular}
\label{Table:Reeber-inputs}
\end{scriptsize}
\end{table*}
