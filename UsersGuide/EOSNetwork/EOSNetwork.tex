\section{Equation of State}
CASTRO is written in a modular fashion so that the EOS and network
burning routines can be supplied by the user.   However, for the
examples presented later we use several EOS and network routines
that come with the CASTRO distribution.  

EOS routines that come with CASTRO are (listed by directory name):
\begin{itemize}
\item {\bf GammaLawEOS} directory represents a gamma law gas.
\item {\bf HelmEOS} directory contains a general, publicly available
stellar equation of state based on the Helmholtz free energy,
with contributions from ions, radiation, and electron degeneracy, as
described in (Timmes and Arnett 1999, Times and Swesty 2000, and Fryxell et al. 2000).
\item {\bf LattimerSwestyEOS} directory contains a modified version of the
LS EOS available at http://www.astro.sunysb.edu/dswesty.  Full
documentation is available through that web site.  We use this EOS
in the 1D core collapse supernova example in a later section.
\end{itemize}

Each EOS directory contains two subroutines by which it interfaces to
the rest of the CASTRO code.  The first, 
\[
{\bf \rm EosGivenRTX}(e^\outp, p^\outp, \rho^\inp, T^\inp, X^\inp, n_{\rm spec}, Y^\inp, n_{\rm aux})
\]
is a direct interface to the EOS, in which density, species and auxiliary variables,
and temperature are specified, and the necessary thermodynamical variables such as 
internal energy, pressure, gamma, and sound speed are returned.

The second routine, 
\[
{\bf \rm EosGivenReX}(\Gamma^\outp, p^\outp, c^\outp, T^\outp, \rho^\inp, e^\inp, X^\inp, n_{\rm spec}, Y^\inp, n_{\rm aux})
\]
uses a Newton iteration to find the temperature given
the internal energy, density, and species and auxiliary variables.  
\section{Burning Network}
Burning network routines that come with CASTRO are (listed by directory name):
\begin{itemize}
\item {\bf networks/null} directory describes a non-reacting white dwarf,
with only hydrogen, helium and carbon12.   There are
no auxiliary variables, and no reactions are allowed.
\item {\bf networks/collapse} directory describes a pre-supernova neutron
star with hydrogen, helium, oxygen and iron, There is one auxiliary
variable, Ye, the electron fraction.  Again no reactions are
allowed.
\item {\bf networks/ignition} directory contains a single-step
$^{12}\mathrm{C}(^{12}\mathrm{C},\gamma)^{24}\mathrm{Mg}$ reaction.
The carbon mass fraction equation appears as
\begin{equation}
\frac{D X(^{12}\mathrm{C})}{Dt} = - \frac{1}{12} \rho X(^{12}\mathrm{C})^2
    f_\mathrm{Coul} \left [N_A \left <\sigma v \right > \right]\enskip,
\end{equation}
where $N_A \left <\sigma v\right>$ is evaluated using the reaction
rate from (Caughlan and Fowler 1988).  The Coulomb screening factor,
$f_\mathrm{Coul}$, is evaluated using the general routine from the
Kepler stellar evolution code (Weaver 1978), which implements
the work of (Graboske 1973) for weak screening and the work of
(Alastuey 1978 and Itoh 1979) for strong screening.
\end{itemize}

There are two primary files within each network directory. The first,
castro\_burner.f90, contains the burner routine, 
which takes $\rho^\inp, e^\inp, X_k^\inp$, and $\Delta t$ as inputs.
It is possible for the internal energy, $e$, which is computed from $\Ub$, to be
negative due to roundoff error.  CASTRO has an option to protect against using a 
negative value of $e$ by recomputing $e = e(\rho,T_{\rm small},X_k)$ using the 
equation of state, where $T_{\rm small}$ is a user-defined temperature floor.  In the 
event that $e$ is still negative, we abort the program.  CASTRO also has an option to
skip the reactions if the density is below a user-defined density floor.

Next, the burner computes $T = T(\rho^\inp,e^\inp,X_k^\inp)$ using the equation of state.
The burner returns $X_k^\outp$ and $e^\outp$ by solving over a time interval of $\Delta t/2$,
\begin{eqnarray}
\frac{\partial X_k}{\partial t} &=& \omegadot_k.\\
\end{eqnarray}
In particular, to evolve the species, we solve the system:
\begin{eqnarray}
\frac{dX_k}{dt} &=& \omegadot_k(\rho,X_k,T)\enskip, \label{eq:VODE1C} \\
\frac{dT}{dt} &=&\frac{1}{c_p} \left ( -\sum_k \xi_k  \omegadot_k  \right )\enskip. \label{eq:tempreactC}
\end{eqnarray}
\MarginPar{Need to include temperature evolution equation somewhere}
using the stiff ordinary differential equation integration methods provided by 
the VODE package.  The absolute error tolerances are set to 
$10^{-12}$ for the species, and a relative tolerance of $10^{-5}$ is used for 
the temperature.  The integration yields the new values of the mass fractions, 
$X_k^\outp$.  Equation (\ref{eq:tempreactC}) is derived from equation (???) by 
assuming that the pressure is constant during the burn state.  In evolving these 
equations, we need to evaluate $c_p$ and $\xi_k$.  In theory, this means 
evaluating the equation of state for each right-hand side evaluation that 
VODE requires.  In practice, we freeze $c_p$ and $\xi_k$ at the start of 
the integration time step and compute them using $\rho^\inp, X_k^\inp,$ and $T^\inp$ 
as inputs to the equation of state.  Note that the density remains unchanged during 
the burning.  At the end of the routine, we compute 
$T^\outp = T(\rho^\outp,e^{\outp},X_k^\outp)$.

The second file, ``network.f90'', supply the
number of species and auxiliary variables, names of each species and 
auxiliary variable, as well as other initializing data, such as
aion, zion and the binding energy.

It is straightforward to implement additional EOS and network routines; all that is required
is to create an appropriate interface to the CASTRO calls, which is easily done given
the prototypes supplied with the CASTRO distribution.
