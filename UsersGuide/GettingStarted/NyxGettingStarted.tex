
\section{Downloading the Code}

\nyx\ is built on top of the \amrex\ framework.  In order to run
\nyx\, you must download two separate git modules.

\vspace{.1in}

\noindent First, make sure that {\tt git} is installed on your machine---we recommend version 1.7.x or higher.

\vspace{.1in}

\begin{enumerate}

\item Clone/fork the \amrex\ repository -- you will need to 
get this from the private LBL bitbucket repository by emailing
ASAlmgren@lbl.gov until the public licensing is complete.

You will want to periodically update \amrex\ by typing
\begin{verbatim}
git pull
\end{verbatim}
in the {\tt amrex/} directory.

Note: active development is done on the {\tt development} branch
in each repo, and merged into the {\tt master} branch periodically.
If you wish to use the \nyx\ {\tt development} branch, then you
should also switch to the {\tt development} branch for \amrex.

\item Set the environment variable, {\tt AMREX\_HOME}, on your
  machine to point to the path name where you have put \amrex.
  You can add this to your {\tt .bashrc} as:
\begin{Verbatim}[commandchars=\\\{\}]
export AMREX_HOME={\em /path/to/amrex/}
\end{Verbatim}
where you replace \texttt{\em /path/to/amrex/} will the full path to the
{\tt amrex/} directory.

\item Clone/fork the \nyx\ repository using either HTTP access:
\begin{verbatim}
git clone https://github.com/BoxLib-Codes/Nyx.git
\end{verbatim}
or SSH access if you have it enabled:
\begin{verbatim}
git clone ssh://git@github.com:/BoxLib-Codes/Nyx.git
\end{verbatim}

As with \amrex, development on \nyx\ is done in the
{\tt development} branch, so you should work there if you want
the latest source.

\end{enumerate}

\section{Building the Code}

\begin{enumerate}

\item From the directory in which you checked out Nyx, type

cd {\tt Nyx/Exec/LyA}

This will put you into a directory in which you can run a small
version of the Santa Barbara test problem.

\item In {\tt Nyx/Exec/LyA}, edit the GNUmakefile, and set

COMP = your favorite compiler (e.g, gnu, Intel)

DEBUG = FALSE

We like COMP = gnu.

\item Now type "make". The resulting executable will look something like 
"Nyx3d.Linux.gnu.ex", which means this is a 3-d version of the code, 
made on a Linux machine, with COMP = gnu.

\end{enumerate}

\section{Running the Code}

\begin{enumerate}

\item Type ``Nyx3d.Linux.gnu.ex inputs.32''

\item You will notice that running the code generates directories that look like 
plt00000, plt00020, etc, and chk00000, chk00020, etc. These are "plotfiles" and 
"checkpoint" files. The plotfiles are used for visualization, the checkpoint files are 
used for restarting the code.

\end{enumerate}

\section{Visualization of the Results}

There are several options for visualizing the data.  The popular
\visit\ package supports the \amrex\ file format natively, as does
the \yt\ python package.  The standard tool used within the
\boxlib-community is \amrvis, which we demonstrate here.

\begin{enumerate}

\item Get \amrvis:

\begin{verbatim}
git clone https://ccse.lbl.gov/pub/Downloads/Amrvis.git
\end{verbatim}

Then cd into {\tt Amrvis/}, edit the {\tt GNUmakefile} there
to set {\tt DIM = 2}, and again set {\tt COMP} and {\tt FCOMP} to compilers that
you have. Leave {\tt DEBUG = FALSE}.

Type {\tt make} to build, resulting in an executable that
looks like {\tt amrvis2d...ex}.

If you want to build amrvis with {\tt DIM = 3}, you must first
download and build {\tt volpack}:
\begin{verbatim}
git clone https://ccse.lbl.gov/pub/Downloads/volpack.git
\end{verbatim}

Then cd into {\tt volpack/} and type {\tt make}.

Note: \amrvis\ requires the OSF/Motif libraries and headers. If you don't have these
you will need to install the development version of motif through your package manager.
{\tt lesstif} gives some functionality and will allow you to build the amrvis executable,
but \amrvis\ may not run properly.

On most Linux distributions, the motif library is provided by the
{\tt openmotif} package, and its header files (like {\tt Xm.h}) are provided
by {\tt openmotif-devel}. If those packages are not installed, then use the
package management tool to install them, which varies from
distribution to distribution, but is straightforward.

You may then want to create an alias to {\tt amrvis2d}, for example
\begin{verbatim}
alias amrvis2d /tmp/Amrvis/amrvis2d...ex
\end{verbatim}
where {\tt /tmp/Amrvis/amrvis2d...ex} is the full path and name of the \amrvis\ executable.

\item Configure \amrvis:

  Copy the {\tt amrvis.defaults} file to your home directory (you can
  rename it to {\tt .amrvis.defaults} if you wish).  Then edit the
  file, and change the {\tt palette} line to point to the full
  path/filename of the {\tt Palette} file that comes with \amrvis.

\item Visualize:

  Return to the {\tt Nyx/Exec/LyA} directory.  You should
  have a number of output files, including some in the form {\tt *pltXXXXX},
  where {\tt XXXXX} is a number corresponding to the timestep the file
  was output.  {\tt
    amrvis2d {\em filename}} to see a single plotfile, or {\tt amrvis2d -a
  *plt*}, which will animate the sequence of plotfiles.

  Try playing
  around with this---you can change which variable you are
  looking at, select a region and click ``Dataset'' (under View)
  in order to look at the actual numbers, etc. You can also export the
  pictures in several different formats under "File/Export".

Please know that we do have a number of conversion routines to other
formats (such as matlab), but it is hard to describe them all. If you
would like to display the data in another format, please let us know
(again, {\tt asalmgren@lbl.gov}) and we will point you to whatever we have
that can help.

\end{enumerate}
