\section{Getting Started With Git}

Please note that Nyx is not yet available for public release.  If you have
heard about Nyx and are interested in trying it, please contact Ann Almgren at
asalmgren@lbl.gov

Nyx is now distributed in two parts using git -- 
you must first download the BoxLib repository, then the Nyx repo.

To use Nyx:

\begin{enumerate}

\item Assuming git is installed on your machine -- and we recommend 
version 1.7.x or higher -- download the BoxLib repository by typing  \\

\noindent git clone https://ccse.lbl.gov/pub/Downloads/BoxLib.git

\noindent This will create a directory called BoxLib on your machine.  Put this somewhere out of the way and 
set the environment variable, {\bf BOXLIB\_HOME}, on your machine to the path name where
you have put BoxLib.    You will want to periodically update BoxLib by typing 

\noindent git pull

in the BoxLib directory.  

\item We will have set up an account for you; follow the instructions 
we have given you to access Nyx itself. 

\end{enumerate}

\clearpage

\section{Building the Code}

\begin{enumerate}

\item From the directory in which you checked out Nyx, type

cd Nyx/Exec/Test\_90Mpc

This will put you into a directory in which you can run the 90Mpc test problem from 
{\tt http://t8web.lanl.gov/people/heitmann/arxiv/}

\item In Test\_90Mpc, edit the GNUmakefile, and set

COMP = your favorite C++ compiler

FCOMP = your favorite Fortran compiler (which must compile F90)

DEBUG = FALSE

We like COMP = gcc and FCOMP = gfortran. 

\item Now type "make". The resulting executable will look something like 
"Nyx3d.Linux.gcc.gfortran.ex", which means this is a 3-d version of the code, 
made on a Linux machine, with COMP = gcc and FCOMP = gfortran.

\end{enumerate}

\section{Running the Code}

\begin{enumerate}

\item Type ``Nyx3d.Linux.gcc.gfortran.ex inputs''

\item You will notice that running the code generates directories that look like 
plt00000, plt00020, etc, and chk00000, chk00020, etc. These are "plotfiles" and 
"checkpoint" files. The plotfiles are used for visualization, the checkpoint files are 
used for restarting the code.

\end{enumerate}

\section{Visualization of the Results}

\begin{enumerate}

\item To visualize the plotfiles, you can use Visit, yt, or 
a homegrown visualization tool known as {\bf amrvis}.

To use {\bf amrvis}, please contact Ann Almgren (asalmgren@lbl.gov) for information on how
to download it.  Visit and yt are available from their web sites.

\end{enumerate}

You have now completed a brief introduction to Nyx. 
