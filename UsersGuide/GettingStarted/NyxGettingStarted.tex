
\section{Introduction}

Welcome to the Nyx User's Guide!  In this document is an overview of Nyx, an adaptive mesh
compressible cosmological hydrodynamics simulation code.
Another place to look for information about what is being done with Nyx is
{\url{https://amrex-astro.github.io/Nyx}} -- definitely check that out for
the latest science being done with Nyx and papers published.

\section{Downloading the Code}

\nyx\ is built on top of the \amrex\ framework.  In order to run
\nyx\, you must download two separate git modules.

\vspace{.1in}

\noindent First, make sure that {\tt git} is installed on your machine---we recommend version 1.7.x or higher.

\vspace{.1in}

\begin{enumerate}

\item Clone/fork the \amrex\ repository:
\begin{verbatim}
git clone https://github.com/AMReX-Codes/amrex
\end{verbatim}

You will want to periodically update \amrex\ by typing
\begin{verbatim}
git pull
\end{verbatim}
in the {\tt amrex/} directory.

Note: when you check out amrex (and Nyx), you will get the {\tt master}
branch.   The Nyx master branch is guaranteed to be compatible
with the amrex master branch.
Active development is done on the {\tt development} branch
in each repo, and merged into the {\tt master} branch monthly.
If you wish to use the \nyx\ {\tt development} branch, then you
should also switch to the {\tt development} branch for \amrex.

\item Set the environment variable, {\tt AMREX\_HOME}, on your
  machine to point to the path name where you have put \amrex.
  You can add this to your {\tt .bashrc} as:
\begin{Verbatim}[commandchars=\\\{\}]
export AMREX_HOME={\em /path/to/amrex/}
\end{Verbatim}
or to your {\tt .cshrc} as:
\begin{Verbatim}[commandchars=\\\{\}]
setenv AMREX_HOME {\em /path/to/amrex/}
\end{Verbatim}
where you replace \texttt{\em /path/to/amrex/} will the full path to the
{\tt amrex/} directory.

\item Clone/fork the \nyx\ repository:
\begin{verbatim}
git clone https://github.com/AMReX-Astro/Nyx
\end{verbatim}

As with \amrex, development on \nyx\ is done in the
{\tt development} branch, so you should work there if you want
the latest source.

\end{enumerate}

\section{Building the Code}

\begin{enumerate}

\item From the directory in which you checked out Nyx, type

cd {\tt Nyx/Exec/LyA}

This will put you into a directory in which you can run a small
version of the Santa Barbara test problem.

\item In {\tt Nyx/Exec/LyA}, edit the GNUmakefile, and set

COMP = your favorite compiler (e.g, gnu, Intel)

DEBUG = FALSE

We like COMP = gnu.

\item Now type "make". The resulting executable will look something like 
"Nyx3d.Linux.gnu.ex", which means this is a 3-d version of the code, 
made on a Linux machine, with COMP = gnu.

\end{enumerate}

\section{Running the Code}

\begin{enumerate}

\item Type ``Nyx3d.Linux.gnu.ex inputs.32''

\item You will notice that running the code generates directories that look like 
plt00000, plt00020, etc, and chk00000, chk00020, etc. These are "plotfiles" and 
"checkpoint" files. The plotfiles are used for visualization, the checkpoint files are 
used for restarting the code.

\end{enumerate}

\section{Visualization}

There are several visualization tools that can be used for Nyx
plotfiles.  The standard tool used within the
\amrex-community is \amrvis, a package developed and supported
by CCSE that is designed specifically for highly efficient visualization
of block-structured hierarchical AMR data.
Plotfiles can also be viewed using the \visit, \paraview, and \yt\ packages.
Particle data can be viewed using \paraview.

Please see Chapter 9 of the AMReX User's Guide (available in amrex/Docs)
for more detail about using all of these visualization packages.
